% Options for packages loaded elsewhere
\PassOptionsToPackage{unicode}{hyperref}
\PassOptionsToPackage{hyphens}{url}
\PassOptionsToPackage{dvipsnames,svgnames,x11names}{xcolor}
%
\documentclass[
  letterpaper,
  DIV=11,
  numbers=noendperiod]{scrartcl}

\usepackage{amsmath,amssymb}
\usepackage{iftex}
\ifPDFTeX
  \usepackage[T1]{fontenc}
  \usepackage[utf8]{inputenc}
  \usepackage{textcomp} % provide euro and other symbols
\else % if luatex or xetex
  \usepackage{unicode-math}
  \defaultfontfeatures{Scale=MatchLowercase}
  \defaultfontfeatures[\rmfamily]{Ligatures=TeX,Scale=1}
\fi
\usepackage{lmodern}
\ifPDFTeX\else  
    % xetex/luatex font selection
\fi
% Use upquote if available, for straight quotes in verbatim environments
\IfFileExists{upquote.sty}{\usepackage{upquote}}{}
\IfFileExists{microtype.sty}{% use microtype if available
  \usepackage[]{microtype}
  \UseMicrotypeSet[protrusion]{basicmath} % disable protrusion for tt fonts
}{}
\makeatletter
\@ifundefined{KOMAClassName}{% if non-KOMA class
  \IfFileExists{parskip.sty}{%
    \usepackage{parskip}
  }{% else
    \setlength{\parindent}{0pt}
    \setlength{\parskip}{6pt plus 2pt minus 1pt}}
}{% if KOMA class
  \KOMAoptions{parskip=half}}
\makeatother
\usepackage{xcolor}
\setlength{\emergencystretch}{3em} % prevent overfull lines
\setcounter{secnumdepth}{-\maxdimen} % remove section numbering
% Make \paragraph and \subparagraph free-standing
\makeatletter
\ifx\paragraph\undefined\else
  \let\oldparagraph\paragraph
  \renewcommand{\paragraph}{
    \@ifstar
      \xxxParagraphStar
      \xxxParagraphNoStar
  }
  \newcommand{\xxxParagraphStar}[1]{\oldparagraph*{#1}\mbox{}}
  \newcommand{\xxxParagraphNoStar}[1]{\oldparagraph{#1}\mbox{}}
\fi
\ifx\subparagraph\undefined\else
  \let\oldsubparagraph\subparagraph
  \renewcommand{\subparagraph}{
    \@ifstar
      \xxxSubParagraphStar
      \xxxSubParagraphNoStar
  }
  \newcommand{\xxxSubParagraphStar}[1]{\oldsubparagraph*{#1}\mbox{}}
  \newcommand{\xxxSubParagraphNoStar}[1]{\oldsubparagraph{#1}\mbox{}}
\fi
\makeatother


\providecommand{\tightlist}{%
  \setlength{\itemsep}{0pt}\setlength{\parskip}{0pt}}\usepackage{longtable,booktabs,array}
\usepackage{calc} % for calculating minipage widths
% Correct order of tables after \paragraph or \subparagraph
\usepackage{etoolbox}
\makeatletter
\patchcmd\longtable{\par}{\if@noskipsec\mbox{}\fi\par}{}{}
\makeatother
% Allow footnotes in longtable head/foot
\IfFileExists{footnotehyper.sty}{\usepackage{footnotehyper}}{\usepackage{footnote}}
\makesavenoteenv{longtable}
\usepackage{graphicx}
\makeatletter
\def\maxwidth{\ifdim\Gin@nat@width>\linewidth\linewidth\else\Gin@nat@width\fi}
\def\maxheight{\ifdim\Gin@nat@height>\textheight\textheight\else\Gin@nat@height\fi}
\makeatother
% Scale images if necessary, so that they will not overflow the page
% margins by default, and it is still possible to overwrite the defaults
% using explicit options in \includegraphics[width, height, ...]{}
\setkeys{Gin}{width=\maxwidth,height=\maxheight,keepaspectratio}
% Set default figure placement to htbp
\makeatletter
\def\fps@figure{htbp}
\makeatother
% definitions for citeproc citations
\NewDocumentCommand\citeproctext{}{}
\NewDocumentCommand\citeproc{mm}{%
  \begingroup\def\citeproctext{#2}\cite{#1}\endgroup}
\makeatletter
 % allow citations to break across lines
 \let\@cite@ofmt\@firstofone
 % avoid brackets around text for \cite:
 \def\@biblabel#1{}
 \def\@cite#1#2{{#1\if@tempswa , #2\fi}}
\makeatother
\newlength{\cslhangindent}
\setlength{\cslhangindent}{1.5em}
\newlength{\csllabelwidth}
\setlength{\csllabelwidth}{3em}
\newenvironment{CSLReferences}[2] % #1 hanging-indent, #2 entry-spacing
 {\begin{list}{}{%
  \setlength{\itemindent}{0pt}
  \setlength{\leftmargin}{0pt}
  \setlength{\parsep}{0pt}
  % turn on hanging indent if param 1 is 1
  \ifodd #1
   \setlength{\leftmargin}{\cslhangindent}
   \setlength{\itemindent}{-1\cslhangindent}
  \fi
  % set entry spacing
  \setlength{\itemsep}{#2\baselineskip}}}
 {\end{list}}
\usepackage{calc}
\newcommand{\CSLBlock}[1]{\hfill\break\parbox[t]{\linewidth}{\strut\ignorespaces#1\strut}}
\newcommand{\CSLLeftMargin}[1]{\parbox[t]{\csllabelwidth}{\strut#1\strut}}
\newcommand{\CSLRightInline}[1]{\parbox[t]{\linewidth - \csllabelwidth}{\strut#1\strut}}
\newcommand{\CSLIndent}[1]{\hspace{\cslhangindent}#1}

\KOMAoption{captions}{tableheading}
\makeatletter
\@ifpackageloaded{caption}{}{\usepackage{caption}}
\AtBeginDocument{%
\ifdefined\contentsname
  \renewcommand*\contentsname{Table of contents}
\else
  \newcommand\contentsname{Table of contents}
\fi
\ifdefined\listfigurename
  \renewcommand*\listfigurename{List of Figures}
\else
  \newcommand\listfigurename{List of Figures}
\fi
\ifdefined\listtablename
  \renewcommand*\listtablename{List of Tables}
\else
  \newcommand\listtablename{List of Tables}
\fi
\ifdefined\figurename
  \renewcommand*\figurename{Figure}
\else
  \newcommand\figurename{Figure}
\fi
\ifdefined\tablename
  \renewcommand*\tablename{Table}
\else
  \newcommand\tablename{Table}
\fi
}
\@ifpackageloaded{float}{}{\usepackage{float}}
\floatstyle{ruled}
\@ifundefined{c@chapter}{\newfloat{codelisting}{h}{lop}}{\newfloat{codelisting}{h}{lop}[chapter]}
\floatname{codelisting}{Listing}
\newcommand*\listoflistings{\listof{codelisting}{List of Listings}}
\makeatother
\makeatletter
\makeatother
\makeatletter
\@ifpackageloaded{caption}{}{\usepackage{caption}}
\@ifpackageloaded{subcaption}{}{\usepackage{subcaption}}
\makeatother

\ifLuaTeX
  \usepackage{selnolig}  % disable illegal ligatures
\fi
\usepackage{bookmark}

\IfFileExists{xurl.sty}{\usepackage{xurl}}{} % add URL line breaks if available
\urlstyle{same} % disable monospaced font for URLs
\hypersetup{
  pdftitle={Regional growth, convergence, and spatial spillovers in India: A view from outer space},
  pdfkeywords={Regional luminosity, Satellite nighttime lights, Regional
convergence, Spatial spillovers, Spatial Durbin model, India},
  colorlinks=true,
  linkcolor={blue},
  filecolor={Maroon},
  citecolor={Blue},
  urlcolor={Blue},
  pdfcreator={LaTeX via pandoc}}


\title{Regional growth, convergence, and spatial spillovers in India: A
view from outer space}
\author{\_ \_}
\date{}

\begin{document}
\maketitle
\begin{abstract}
Using satellite nighttime light data as proxy for economic activity,
Chanda and Kabiraj (2020, World Development) studied regional growth and
convergence across 520 districts in India. This article builds on their
work by confirming and extending their main findings in three fronts.
First, we illustrate regional convergence patterns using an interactive
web-based tool for satellite image visualization. Second, we assess the
degree of spatial dependence in their main econometric specification.
Third, we employ a spatial Durbin model to measure the role of spatial
spillovers in the convergence process. Our results indicate that spatial
spillovers significantly increase the speed of regional convergence.
Overall, the results emphasize the role of spatial dependence in
regional convergence through the lens of satellite imagery, interactive
visualizations, and spillover modeling.
\end{abstract}


\subsection{Introduction}\label{introduction}

Regional economic growth and convergence are key concerns in developing
countries, particularly in large federal states like India where spatial
inequalities can threaten social cohesion and political stability.
However, studying regional convergence patterns in developing countries
has been historically challenging due to limited availability of
consistent economic data at subnational administrative levels. The
emergence of satellite nighttime light data as a proxy for economic
activity has created new opportunities to analyze regional growth
dynamics at granular geographic scales.

In an important contribution, Chanda and Kabiraj (2020) leveraged
nighttime light data to document evidence of regional convergence across
520 districts in India between 1996 and 2010. Their analysis showed that
poorer districts grew faster than richer ones during this period,
suggesting a gradual reduction in spatial inequalities. However, their
econometric approach did not account for potential spatial spillovers in
the convergence process---the possibility that a district's growth
trajectory might be influenced not only by its own initial conditions
but also by those of its neighbors.

In this context, this paper confirms and extends the study of Chanda and
Kabiraj (2020) in three key methodological directions. First, we develop
an interactive web-based visualization tool that allows researchers to
dynamically explore spatial and temporal patterns in the nighttime light
data. This tool facilitates the identification of converging regions and
growth hotspots that may be difficult to detect in static
representations. Second, we formally test for spatial dependence in both
the dependent and independent variables of the convergence equations,
highlighting that spatial autocorrelation is an inherent feature of
satellite data and the regional convergence process. Third, we employ a
spatial Durbin model that explicitly accounts for spatial spillovers,
quantifying how neighborhood effects influence the speed of regional
convergence.

Our results yield three main findings that advance our understanding of
regional convergence in India. First, interactive visualization tools
reveal clear spatial patterns in both the initial distribution and
subsequent growth of nighttime lights. Second, formal tests of spatial
dependence indicate that district-level economic trajectories are not
independent of their neighbors. Third, accounting for spatial spillovers
through the spatial Durbin model shows that the total convergence effect
is substantially larger than previous non-spatial estimates would
suggest. Specifically, spatial spillovers appear to accelerate the
convergence process by creating additional channels through which
lagging regions can catch up.

These findings have important implications for both research methodology
and policy design. Methodologically, they demonstrate that conventional
non-spatial approaches may significantly underestimate the speed of
regional convergence by failing to account for inter-district
spillovers. From a policy perspective, they suggest that the benefits of
place-based development interventions may extend beyond target districts
through spatial multiplier effects, potentially increasing their
cost-effectiveness. The results also highlight the value of new data
sources and methodological tools in advancing our understanding of
regional economic dynamics in developing countries.

The rest of this article is organized as follows. Section 2 provides an
overview of the data and methods, describing our use of nighttime light
data as a proxy for economic activity and introducing the spatial Durbin
model that forms the basis of our empirical strategy. We also detail our
methodological extensions related to interactive visualizations, spatial
dependence testing, and spillover modeling. Section 3 presents our
empirical results, beginning with an interactive exploration of regional
convergence patterns, followed by formal tests of spatial dependence,
and concluding with estimates of direct and indirect convergence effects
from the spatial Durbin model. Finally, Section 4 offers some concluding
remarks.

\subsection{Concluding remarks}\label{concluding-remarks}

This article examines regional convergence patterns across Indian
districts using satellite nighttime light data, interactive
visualizations, and spatial econometric modeling. Expanding on the work
of Chanda and Kabiraj (2020), we developed an interactive web-based
visualization tool that illustrates spatial and convergence patterns
across Indian districts. Spatial autocorrelation tests confirm that
spatial dependence is a fundamental characteristic of satellite data and
the regional convergence process in India. Estimates from our spatial
Durbin model indicate that incorporating spatial spillovers
significantly increases the estimated speed of regional convergence. The
total convergence effect in our fully specified model is approximately
36\% larger than conventional non-spatial estimates of (Chanda and
Kabiraj 2020). This finding implies that non-spatial convergence models
may considerably underestimate the speed of regional convergence.
Additionally, it suggests that place-based development interventions may
have broader impacts, as their benefits can extend to neighboring
districts through spatial spillover effects.

These results have important implications for both research methodology
and policy design in developing countries. From a methodological
perspective, they highlight the value of combining new data sources,
interactive geospatial visualization tools, and spatial econometric
methods to better understand regional economic dynamics. From a policy
standpoint, they suggest that regional convergence operates not just
through district-specific factors, but through complex spatial
interactions that create additional channels for catch-up growth. This
spatial perspective is crucial for the analysis and design of policies
aimed at reducing regional inequalities in developing countries.

\phantomsection\label{refs}
\begin{CSLReferences}{1}{0}
\bibitem[\citeproctext]{ref-chanda2020}
Chanda, Areendam, and Sujana Kabiraj. 2020. {``Shedding Light on
Regional Growth and Convergence in India.''} \emph{World Development}
133: 104961.

\end{CSLReferences}




\end{document}
